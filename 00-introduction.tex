\chapter*{Introduction}
\addcontentsline{toc}{chapter}{Introduction}

Unitex is a collection of programs developped for the analysis of texts in
natural language by using linguistic resources and tools. These resources
consist of electronic dictionaries, grammars and lexicon-grammar tables,
initially developed  for French by Maurice Gross and his students  at the
Laboratoire d'Automatique Documentaire et Linguistique (LADL).\index{LADL}
Similar resources have been developed for other languages in the context of the
RELEX laboratory network.

\bigskip
\noindent The electronic dictionaries specify the simple and compound words of a
language together with their lemmas and a set of grammatical (semantic and inflectional)
codes. The availability of these dictionaries is a major advantage compared to
the usual utilities for pattern searching as the information they contain can be
used  for searching and matching,  thus  describing large classes of words using
very simple patterns. The dictionaries are presented in the DELA formalism and
were constructed  by teams of  linguists for several languages (French, English,
Greek, Italian, Spanish, German, Thai, Korean, Polish, Norwegian, Portuguese,
etc.)

\bigskip
\noindent The grammars used here are representations of linguistic phenomena on the
basis of  recursive transition networks (RTN), a formalism closely related to
finite state automata. Numerous studies have shown the adequacy of automata for
linguistic problems at all descriptive levels  from morphology and syntax to
phonetic issues. Grammars created with Unitex carry this approach further  by
using a formalism even more powerful than automata. These grammars are
represented as graphs that the user can easily create and update.

\bigskip
\noindent Lexicon-grammar tables are matrices describing
properties of some words. Many such tables have been constructed   for all
simple verbs in French as a way of describing their relevant
syntactic properties. Experience has shown that every word has a
quasi-unique behavior, and these tables are a way to present the 
grammar of every element in the lexicon, hence the name lexicon-grammar 
for this linguistic theory. Unitex offers a way to
automatically build grammars from lexicon-grammar tables.

\bigskip
\noindent Unitex can be viewed as a tool in which one can put linguistic resources
and use them. Its technical characteristics are its portability,  modularity,
the possibility of dealing with languages that use special writing systems (e.g. many
Asian languages), and its openness, thanks to its open source distribution. Its
linguistic characteristics are the ones that have motivated the elaboration of
these resources: precision, completeness, and the taking into account of frozen
expressions, most notably those which concern the enumeration of compound words.


\section*{What's new from version 1.2 ?}
\addcontentsline{toc}{section}{What's new from version 1.2 ?}
Here are some interesting new features:
\begin{itemize}
  \item left contexts
  \item morphological mode in \verb$Locate$
  \item brand new version of \verb$Convert$
  \item replacement of \verb$Inflect$ by \verb$MultiFlex$ that can inflect compound words
  and that can handle consonant skeletons for semitic languages
  \item introduction of the text alignment tool \verb$XAlign$
  \item no size limit for text file display
  \item SVG export of graphs
\end{itemize}

\bigskip \noindent From a computational point of view, a special effort has been
made to clean and comment the source code of Unitex programs in order to
facilitate the integration of new components. Moreover, the development of Unitex
is now made with a SVN server, which makes collaborative work much more easier.


\section*{Content}
\addcontentsline{toc}{section}{Content}
\noindent Chapter \ref{chap-installation} describes how to install and run
Unitex.

\bigskip \noindent Chapter \ref{chap-texte} presents the different steps in the
analysis of  a text.

\bigskip \noindent Chapter \ref{chap-dictionaries} describes the formalism of
the DELA electronic dictionaries and the different operations that can be applied to them.

\bigskip \noindent Chapters \ref{chap-regexp} and \ref{chap-grammaires-locales}
present different means for making text searches more effective. 
Chapter \ref{chap-grammaires-locales} describes in detail how to use the graph
editor.

\bigskip \noindent Chapter \ref{chap-grammaires-avanc�es} is concerned with the
different possible applications of grammars. The particularities of each type of grammar are
presented.

\bigskip \noindent Chapter \ref{chap-automate-du-texte} introduces the  concept
of text automaton and describes the properties of this notion. This chapter also describes 
operations on this object, in particular, how to disambiguate lexical items with
the ELAG program.

\bigskip \noindent Chapter \ref{chap-lexique-grammaire} contains an introduction
to lexicon-grammar tables, followed by a description of the method of constructing grammars based on these
tables.

\bigskip \noindent Chapter \ref{chap-alignment} describes the text
alignment module, based on the XAlign tool.

\bigskip \noindent Chapter \ref{chap-multiflex} describes the compound word
inflection module, as a complement of the simple word inflection mechanism
presented in chapter \ref{chap-dictionaries}.

\bigskip \noindent Chapter \ref{chap-programmes-externes} contains a detailed
description of the external programs that make up the Unitex system.

\bigskip \noindent Chapter \ref{chap-formats-de-fichiers} contains descriptions
of all file formats used in the system.


\bigskip \noindent The reader will find in appendix the GPL and LGPL licenses
under which the Unitex source code is released, as well as the LGPLLR license
which applies for the linguistic data distributed with Unitex.
