\chapter{Utilisation d'Unitex/GramLab à l'aide de scripts}
\label{chap-scripts}\index{Script de programmes Unitex}

On peut utiliser Unitex/GramLab par l'intermédiaire de scripts au lieu de l'interface graphique.
Le script lance des programmes externes d'Unitex, documentés dans le
chapitre~\ref{chap-external-programs}. L'avantage de cette possibilité est qu'on a accès à des options
supplémentaires dans les programmes, par exemple l'option \verb$standoff$ de CasSys
(section~\ref{section-standOff}), et même à des programmes supplémentaires, comme
\verb$DumpOffsets$ (section~\ref{section-DumpOffsets}). On peut aussi, pendant le prétraitement,
enchainer en cascade plusieurs graphes en mode MERGE, alors que l'interface graphique ne permet d'en
lancer qu'un.
L'utilisation de scripts nécessite plus d'habileté en informatique, et en particulier plus de familiarité
avec le système d'exploitation.



\section{Traduire en script un traitement lancé via l'interface graphique}
\label{section-console}

Une façon simple d'écrire un script qui utilise Unitex/GramLab est de lancer l'interface graphique Unitex,
d'y réaliser une première version du traitement, puis de traduire les clics en une suite de commandes.
Pour cela, on peut produire un fichier de journal des opérations lancées (section~\ref{section-log-file}).
On peut aussi tirer parti du fait que l'interface graphique d'Unitex garde en mémoire ces opérations
et peut les afficher dans la
console ("Info > Console", section \ref{section-console}, fig. \ref{fig-console}). On sélectionne les étapes
à retenir, on les copie dans un fichier texte et éventuellement on les adapte en un script opérationnel.
Pour sélectionner des étapes dans la console, cliquer sur les cellules correspondantes dans la colonne
"Log \#" (un majuscule-clic et un ctrl-clic ont l'effet habituel pour sélectionner plusieurs éléments).
On peut ajouter dans le script des programmes ou options documentés dans le
chapitre~\ref{chap-external-programs}, même s'ils ne sont pas disponibles via l'interface.

\bigskip
\noindent On peut formaliser un script sous la forme shell ou batch (section~\ref{section-batch}),
ou encore sous la forme d'un script interprété par Unitex (section~\ref{section-unitex-scripts}).



\section{Scripts shell ou batch}
\label{section-batch}

On peut mettre un script sous la forme shell ou batch et le faire exécuter par l'interpréteur de
commandes du système d'exploitation. Par exemple, les commandes suivantes ouvrent le corpus
\verb$80jours$, sans le prétraitement en mode REPLACE ni en mode MERGE, mais avec le dictionnaire
par défaut, \verb$Dela_fr$, puis lancent un "Locate pattern" avec le graphe \verb$pattern.fst2$
et créent une concordance:

\begin{Verbatim}[fontsize=\small,fontfamily=helvetica]
mkdir "U:\Unitex\French\Corpus\80jours_snt" 
"C:\Unitex-GramLab\App\UnitexToolLogger.exe" Normalize
    "U:\Unitex\French/corpus.txt" "-rU:\Unitex\French\Norm.txt"
    "--output_offsets=U:\Unitex\French\Corpus\80jours_snt\normalize.out.offsets" -qutf8-no-bom
"C:\Unitex-GramLab\App\UnitexToolLogger.exe" Tokenize "U:\Unitex\French\Corpus\80jours.snt" 
    "-aU:\Unitex\French\Alphabet.txt" -qutf8-no-bom
"C:\Unitex-GramLab\App\UnitexToolLogger.exe" Dico "-tU:\Unitex\French\Corpus\80jours.snt" 
    "-aU:\Unitex\French\Alphabet.txt" "C:\Unitex-GramLab\French\Dela\Dela_fr.bin" -qutf8-no-bom
"C:\Unitex-GramLab\App\UnitexToolLogger.exe" SortTxt "U:\Unitex\French\Corpus\80jours_snt\dlf" 
    "-lU:\Unitex\French\Corpus\80jours_snt\dlf.n" "-oU:\Unitex\French\Alphabet_sort.txt"
    -qutf8-no-bom
"C:\Unitex-GramLab\App\UnitexToolLogger.exe" SortTxt "U:\Unitex\French\Corpus\80jours_snt\dlc" 
    "-lU:\Unitex\French\Corpus\80jours_snt\dlc.n" "-oU:\Unitex\French\Alphabet_sort.txt"
    -qutf8-no-bom
"C:\Unitex-GramLab\App\UnitexToolLogger.exe" SortTxt "U:\Unitex\French\Corpus\80jours_snt\err" 
    "-lU:\Unitex\French\Corpus\80jours_snt\err.n" "-oU:\Unitex\French\Alphabet_sort.txt"
    -qutf8-no-bom
"C:\Unitex-GramLab\App\UnitexToolLogger.exe" SortTxt
    "U:\Unitex\French\Corpus\80jours_snt\tags_err"
    "-lU:\Unitex\French\Corpus\80jours_snt\tags_err.n" "-oU:\Unitex\French\Alphabet_sort.txt"
    -qutf8-no-bom
"C:\Unitex-GramLab\App\UnitexToolLogger.exe" Locate "-tU:\Unitex\French\Corpus\80jours.snt" 
    "U:\Unitex\French\Graphs\pattern.fst2" "-aU:\Unitex\French\Alphabet.txt" -L -M --all -b -Y
    --stack_max=1000 --max_matches_per_subgraph=200 --max_matches_at_token_pos=400 
    --max_errors=50 -qutf8-no-bom
"C:\Unitex-GramLab\App\UnitexToolLogger.exe" Concord
    "U:\Unitex\French\Corpus\80jours_snt\concord.ind" "-fCourier new" -s10 -l40 -r55 --html 
    "-aU:\Unitex\French\Alphabet_sort.txt" --CL -qutf8-no-bom
\end{Verbatim}

\noindent Pour les détails des scripts et les commentaires éventuels, il faut respecter les conventions
propres au système d'exploitation.



\section{Scripts interprétés par Unitex/GramLab}
\label{section-unitex-scripts}\index{Package linguistique}

Unitex/GramLab a aussi un interpréteur de scripts, qui a plusieurs avantages par rapport à des scripts
shell ou batch.
\begin{itemize}
    \item L'exécution est plus rapide que celle d'un script shell ou batch équivalent, car l'interpréteur fait
appel à un système de fichiers virtuel.
    \hyphenation{pac-kage}
    \item Le script est encapsulé, avec toutes les ressources qu'il utilise, dans un "package linguistique"
qu'on peut ensuite déployer dans un autre environnement. De cette façon, la mise au point du script
et son utilisation dans une chaine de traitement sont entièrement indépendantes:
\begin{itemize}
    \item on peut déployer le package sans aucune connaissance sur Unitex ni sur les traitements
linguistiques réalisés par le script,
    \item l'environnement de mise au point et l'environnement d'utilisation ne sont pas nécessairement
dans le même système d'exploitation,
    \item le traitement ne modifie aucune donnée dans l'environnement d'utilisation, car il est réalisé
sur une copie du corpus et avec une copie des ressources contenues dans le package. Seuls les fichiers
explicitement spécifiés comme fichiers résultats du traitement sont copiés dans l'environnement
d'utilisation à la fin du traitement.
\end{itemize}
\end{itemize}

\noindent Le langage de script reconnu est constitué des programmes externes d'Unitex, y compris
quelques programmes qui ont été implémentés spécialement pour l'interpréteur de scripts
(section~\ref{section-script-for-runscript}). Il peut y avoir des lignes de commentaires, commençant
par le caractère \#. Les commandes du système d'exploitation ne sont pas reconnues.

\bigskip\index{Programmes externes!\verbc{RunScript}}\index{Programmes externes!\verbc{BatchRunScript}}
\noindent L'interpréteur de scripts d'Unitex/GramLab existe en deux variantes, \verb$RunScript$ et
\verb$BatchRunScript$. \verb$RunScript$ permet de lancer un script sur un corpus contenu dans un
ou plusieurs fichiers. \verb$BatchRunScript$ permet de traiter un corpus constitué par tous les fichiers
d'un répertoire, en lançant le script une fois sur chaque fichier; il produit à chaque fois un fichier résultat
séparé.



\subsection{Réalisation d'un package linguistique}
\label{section-packaga-creation}

Un package linguistique pour Unitex/GramLab regroupe un script et toutes les ressources linguistiques
qu'il utilise, sauf le corpus d'entrée. Le package est organisé sous la forme d'une arborescence
de répertoires, puis compressé au format Zip.

\bigskip     \hyphenation{lan-gue}
\noindent La racine de l'arborescence est un répertoire \verb$<ling_pkg_name>$ avec deux
sous-répertoires, obligatoirement nommés \verb$script$ et \verb$resource$. Le script doit être placé
dans \verb$<ling_pkg_name>/script$ et toutes les ressources linguistiques nécessaires dans
\verb$<ling_pkg_name>/resource$. On peut reproduire dans \verb$resource$ l'organisation d'un
répertoire personnel Unitex, avec un sous-répertoire pour chaque langue traitée par le script
(par exemple \verb$French$), puis, à l'intérieur de chaque répertoire de langue, les fichiers d'alphabet
\verb$Alphabet.txt$ et \verb$Alphabet_sort.txt$, le fichier de normalisation \verb$Norm.txt$, etc. et
des sous-répertoires \verb$Dela$, \verb$Graphs$, etc. contenant à leur tour les ressources
correspondantes. Avec cette organisation, on peut plus facilement écrire le script en adaptant un script
issu de la console.

\bigskip
\noindent Si on souhaite que le script ait accès à la date du package, il faut créer dans le
répertoire \verb$<ling_pkg_name>/resource$ un fichier texte vide dont le nom est
\verb$VERSION_AAAAMMJJ$ (sans extension).

\bigskip
\noindent Une fois cette arborescence réalisée, on la compresse au format Zip.

\bigskip
\noindent Par exemple, si on veut encapsuler dans un package \verb$pkg.zip$ le traitement de la
section~\ref{section-batch}, il faut prévoir dans un répertoire \verb$pkg$:
\begin{itemize}
\item dans le sous-répertoire \verb$resource$:
\begin{itemize}
\item les fichiers d'alphabet, \verb$Alphabet.txt$ et \verb$Alphabet_sort.txt$, et le fichier de
normalisation \verb$Norm.txt$ dans \verb$French$;
\item les fichiers dictionnaire \verb$Dela_fr.bin$ et \verb$Dela_fr.inf$ dans le répertoire
\verb$French/Dela$;
\item le graphe \verb$pattern.fst2$ dans \verb$French/Graphs$ (on peut aussi y placer optionnellement
le fichier \verb$pattern.grf$);
\end{itemize}
\item dans le sous-répertoire \verb$script$, le fichier contenant le script.
\end{itemize}



\subsection{Lancement avec RunScript}
\label{section-runscript}\index{Programmes externes!\verbc{RunScript}}
\index{Programmes externes!\verbc{UnitexToolLogger}}
\verb$RunScript$ permet de lancer un script sur un corpus contenu dans un ou plusieurs fichiers.
Pour cela, on invoque le programme \verb$UnitexToolLogger$ (section~\ref{section-UnitexToolLogger})
en respectant la syntaxe suivante:
\index{Programmes externes!\verbc{SelectOutput}}
\index{Programmes externes!\verbc{InstallLingResourcePackage}}

\begin{Verbatim}[fontsize=\small,fontfamily=helvetica]
UnitexToolLogger [ { SelectOutput <args1> } ] { InstallLingResourcePackage <args2> } { RunScript
    <args3> } { InstallLingResourcePackage <args4> }
\end{Verbatim}

\noindent Chaque paire d'accolades délimite l'invocation d'une commande Unitex avec ses arguments.

\begin{enumerate}
\item L'étape \verb$SelectOutput$ est facultative mais recommandée. En l'incluant sous la forme
suivante:\\
\verb$     { SelectOutput --output=off }$\\
on masque les sorties de mise au point de chaque commande ultérieure, qui sont encombrantes
dans le cas d'une exécution massive.
\item L'étape suivante, \verb$InstallLingResourcePackage$, installe le contenu du package linguistique
dans le système de fichiers virtuel.
\item L'étape \verb$Rusncript$ lance le script. Elle doit définir dans  \verb$INPUT_FILE_1$ et
éventuellement \verb$INPUT_FILE_2$, etc., le nom du ou des fichiers contenant le corpus d'entrée; dans
\verb$OUTPUT_FILE_1$, \verb$OUTPUT_FILE_2$, etc., le nom du ou des fichiers résultats;
dans \verb$PACKAGE_DIR$ le nom du répertoire racine du package linguistique dans le système
de fichiers virtuel; dans \verb$CORPUS_WORK_DIR$ le nom du répertoire qui contiendra les fichiers
créés pendant l'exécution du script.
\item La dernière étape écrit le ou les fichiers résultats, puis met fin au système de fichiers virtuel.
\end{enumerate}

\noindent Par exemple, la commande suivante utilise le package linguistique \verb$pkg.zip$ de la
section~\ref{section-packaga-creation} et lance
le script \verb$uniscript$ contenu dans ce package, avec comme corpus d'entrée \verb$80jours.txt$
et comme fichier de sortie \verb$80jours_result.html$.

\begin{Verbatim}[fontsize=\small,fontfamily=helvetica]
"C:\Unitex-GramLab\App\UnitexToolLogger.exe" { SelectOutput --output=off } {
    InstallLingResourcePackage -p U:\Scripts\pkg.zip -x "$:UnitexPkgResource" -v } { RunScript -v
    -a INPUT_FILE_1=U:\Unitex\French\Corpus\80jours.txt 
    -a "CORPUS_WORK_DIR=$:UnitexPkgWork" -a "PACKAGE_DIR=$:UnitexPkgResource"
    -a OUTPUT_FILE_1=U:\Unitex\French\Corpus\80jours_result.html 
    "$:UnitexPkgResource\script\uniscript" } { InstallLingResourcePackage -p U:\Scripts\pkg.zip 
    -x "$:UnitexPkgResource" -u -v }
\end{Verbatim}



\subsection{Lancement avec BatchRunScript}
\label{launch-script}\index{Programmes externes!\verbc{BatchRunScript}}

\verb$BatchRunScript$ permet de traiter un corpus constitué par tous les fichiers d'un répertoire.
Il lance le script une fois sur chaque fichier, produisant à chaque fois un fichier résultat séparé.
On peut ainsi réaliser des traitements massifs efficacement, même sur des documents nombreux
contenus dans des fichiers séparés. Le résultat est stocké dans un répertoire de sortie, dans des
fichiers portant les mêmes noms que les fichiers d'entrée, suffixés par \verb$.result.txt$.
\verb$BatchRunScript$ fonctionne en multithread et lance plusieurs exécutions en parallèle: ainsi
tous les cœurs de la machine de traitement peuvent fonctionner en même temps et traiter
l’ensemble des fichiers plus vite.

\bigskip
\noindent Pour déployer un script avec \verb$BatchRunScript$, on respecte la syntaxe suivante:

\begin{verbatim}
UnitexToolLogger { BatchRunScript <args> }
\end{verbatim}

\noindent Les arguments doivent définir le nom du répertoire contenant le corpus d'entrée
dans le système de fichiers virtuel, le nom du répertoire de sortie, le nombre de threads, le nom
du package linguistique et le nom du script. Par exemple, la commande suivante traite tous les
fichiers du répertoire \verb$input_folder$, stocke les résultats dans \verb$output_folder$
et fonctionne sur quatre threads:

\begin{Verbatim}[fontsize=\small,fontfamily=helvetica]
"C:\Unitex-GramLab\App\UnitexToolLogger.exe" { BatchRunScript -i .\input_folder -o .\output_folder 
    -t 4 .\pkg -v -p -m -s script/uniscript }
\end{Verbatim}

\noindent Les options \verb$-v$, \verb$-m$ et \verb$-p$ permettent de contrôler quelles 
informations sont émises sur le terminal pendant les traitements. Ainsi, en utilisant \verb$-v -p$,
on affiche le maximum de sorties, ce qui est utile pendant la mise au point. Avec \verb$-m$
et plus encore \verb$-f$, on affiche le minimum de messages.

\bigskip
\noindent Avec \verb$BatchRunScript$, une exécution du script sur un fichier ne peut pas
renvoyer plusieurs fichiers, contrairement à \verb$RunScript$\footnote{Il est possible de
contourner cette limitation avec les outils \texttt{PackFile}  et \texttt{UnpackFile}, en
regroupant plusieurs fichiers résultats en un seul fichier de type Zip.}.



\subsection{Mise au point d'un script pour RunScript}
\label{section-script-for-runscript}
Pour écrire un script dans un package linguistique, il est recommandé d'adapter un script issu
de la console, comme celui de la section~\ref{section-batch}.

\begin{itemize}     \hyphenation{exem-ple}
\item Dans chaque ligne du script issu de la console, supprimer la mention du programme
\verb$UnitexToolLogger$ en début de ligne, c'est-à-dire dans notre exemple
\verb$"C:\Unitex-GramLab\App\UnitexToolLogger.exe"$.
\item Dans les noms des ressources, remplacer le chemin du répertoire de travail par
\verb${PACKAGE_DIR}/resource$. Dans notre exemple, le fichier de normalisation pour le français
devient \verb${PACKAGE_DIR}/resource/French/Norm.txt$.
\item Dans les noms des fichiers créés par le script, remplacer le chemin complet par
\verb${CURRENT_WORK_DIR}$. Dans notre exemple, le nom du fichier \verb$80jours.snt$ devient
\verb${CURRENT_WORK_DIR}/80jours.snt$.
\end{itemize}

\noindent La valeur de \verb$PACKAGEDIR$ est déjà définie quand le script commence à s'exécuter,
mais pour \verb$CURRENT_WORK_DIR$ il faut commencer le script par quelques lignes qui donnent un
nom à ce répertoire, le créent, y copient le corpus d'entrée et font quelques autres préparatifs:

\begin{Verbatim}[fontsize=\small,fontfamily=helvetica]
CURRENT_WORK_DIR = {CORPUS_WORK_DIR}/{UNIQUE_VALUE}
\end{Verbatim}

\noindent Cette ligne donne un nom à \verb$CURRENT_WORK_DIR$. La valeur de
\verb$CORPUS_WORK_DIR$ est déjà définie quand le script commence à s'exécuter. La variable
\verb$UNIQUE_VALUE$ contient une chaine de caractères définie par \verb$RunScript$ ou par
\verb$BatchRunScript$, avec la garantie que la chaine est différente pour chaque exécution de
\verb$RunScript$ ou pour chaque thread de \verb$BatchRunScript$ s’exécutant simultanément. Cela
permet d'éviter toute collision entre fichiers temporaires pendant des travaux simultanés. Ensuite,
comme la commande \verb$mkdir$ n’est pas utilisable dans un script interprété par Unitex/GramLab, on
utilise \verb$DuplicateFile -p$ pour créer le répertoire:
\index{Programmes externes!\verbc{DuplicateFile}}

\begin{Verbatim}[fontsize=\small,fontfamily=helvetica]
DuplicateFile -p {CURRENT_WORK_DIR}
\end{Verbatim}

\noindent On y copie le corpus d’entrée:

\begin{Verbatim}[fontsize=\small,fontfamily=helvetica]
DuplicateFile -i {INPUT_FILE_1} {CURRENT_WORK_DIR}/corpus.txt
\end{Verbatim}

\noindent Dans notre exemple, la première commande du script issu de la console utilisait la commande
\verb$mkdir$ pour créer le sous-répertoire \verb$corpus_snt$. On doit donc la remplacer par
\verb$DuplicateFile$:

\begin{Verbatim}[fontsize=\small,fontfamily=helvetica]
DuplicateFile --make-dir {CURRENT_WORK_DIR}/corpus_snt
\end{Verbatim}

\noindent Les deux lignes suivantes écrivent dans des fichiers des informations sur la version d'Unitex
utilisée:\index{Programmes externes!\verbc{VersionInfo}}


\begin{Verbatim}[fontsize=\small,fontfamily=helvetica]
VersionInfo -n -o {CURRENT_WORK_DIR}/newrevision.txt
VersionInfo -s -o {CURRENT_WORK_DIR}/semver.txt
\end{Verbatim}

\noindent La ligne suivante sauvegarde la date de version du package linguistique, à condition d'avoir
inclus dans le package un fichier \verb$VERSION_AAAAMMJJ$ (voir section \ref{section-packaga-creation}):

\begin{Verbatim}[fontsize=\small,fontfamily=helvetica]
VersionInfo -B -o {CURRENT_WORK_DIR}/builddate.txt
\end{Verbatim}

\noindent Le corps du script peut apparaitre ensuite:

\begin{Verbatim}[fontsize=\small,fontfamily=helvetica]
Normalize "{CURRENT_WORK_DIR}/corpus.txt" "-r{PACKAGE_DIR}/resource/French/Norm.txt" 
    "--output_offsets={CURRENT_WORK_DIR}/corpus_snt/normalize.out.offsets" -qutf8-no-bom
Tokenize "{CURRENT_WORK_DIR}/corpus.snt" "-a{PACKAGE_DIR}/resource/French/Alphabet.txt"
    -qutf8-no-bom
Dico "-t{CURRENT_WORK_DIR}/corpus.snt" "-a{PACKAGE_DIR}/resource/French/Alphabet.txt" 
    "{PACKAGE_DIR}/resource/French/Dela/Dela_fr.bin" -qutf8-no-bom
SortTxt "{CURRENT_WORK_DIR}/corpus_snt/dlf" "-l{CURRENT_WORK_DIR}/corpus_snt/dlf.n" 
    "-o{PACKAGE_DIR}/resource/French/Alphabet_sort.txt" -qutf8-no-bom
SortTxt "{CURRENT_WORK_DIR}/corpus_snt/dlc" "-l{CURRENT_WORK_DIR}/corpus_snt/dlc.n" 
    "-o{PACKAGE_DIR}/resource/French/Alphabet_sort.txt" -qutf8-no-bom
SortTxt "{CURRENT_WORK_DIR}/corpus_snt/err" "-l{CURRENT_WORK_DIR}/corpus_snt/err.n" 
    "-o{PACKAGE_DIR}/resource/French/Alphabet_sort.txt" -qutf8-no-bom
SortTxt "{CURRENT_WORK_DIR}/corpus_snt/tags_err"
    "-l{CURRENT_WORK_DIR}/corpus_snt/tags_err.n" 
    "-o{PACKAGE_DIR}/resource/French/Alphabet_sort.txt" -qutf8-no-bom
Locate "-t{CURRENT_WORK_DIR}/corpus.snt"
    "{PACKAGE_DIR}/resource/French/Graphs/test.fst2" 
    "-a{PACKAGE_DIR}/resource/French/Alphabet.txt" -L -M --all -b -Y --stack_max=1000
    --max_matches_per_subgraph=200 --max_matches_at_token_pos=400 --max_errors=50
    -qutf8-no-bom
Concord "{CURRENT_WORK_DIR}/corpus_snt/concord.ind" "-fCourier new" -s10 -l40 -r55 --html 
    "-a{PACKAGE_DIR}/resource/French/Alphabet_sort.txt" --CL -qutf8-no-bom
\end{Verbatim}

\noindent La fin du script doit indiquer le nom du ou des fichiers résultats et libérer la mémoire:
\index{Programmes externes!\verbc{DuplicateFile}}

\begin{Verbatim}[fontsize=\small,fontfamily=helvetica]
DuplicateFile -i {CURRENT_WORK_DIR}/corpus_snt/concord.html {OUTPUT_FILE_1}
DuplicateFile --recursive-delete {CURRENT_WORK_DIR}
\end{Verbatim}

\noindent On peut lancer le script avec la commande de la section~\ref{launch-script}. Le résultat
obtenu est juste un fichier de concordance \verb$80jours_result.html$, placé à côté de
\verb$80jours.txt$.

\bigskip
\noindent Si on ajoute au script, en avant-dernière position, la ligne:

\begin{Verbatim}[fontsize=\small,fontfamily=helvetica]
DuplicateFile -i {CURRENT_WORK_DIR}/corpus_snt/dlf {OUTPUT_FILE_2}
\end{Verbatim}

\noindent on doit aussi ajouter à la commande de lancement l'option:

\begin{Verbatim}[fontsize=\small,fontfamily=helvetica]
-a OUTPUT_FILE_2=U:\Unitex\French\Corpus\80jours_result.dic
\end{Verbatim}

\noindent On obtient alors comme résultat un fichier dictionnaire \verb$80jours_result.dic$, placé à côté
des deux autres.
