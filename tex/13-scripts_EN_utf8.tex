\chapter{Using Unitex/GramLab with scripts}
\label{chap-scripts}\index{Scripting Unitex programs}

Unitex/GramLab can be used through scripts instead of the graphical interface.
The script launches external programs, which are documented in
Chapter~\ref{chap-external-programs}. The advantage of this possibility is that you may
access additional options in the programs, like option \verb$standoff$ of CasSys
(Section~\ref{section-standOff}), and even additional programs, such as
\verb$DumpOffsets$ (Section~\ref{section-DumpOffsets}). You may also, during preprocessing,
chain graphs in MERGE mode, whereas the graphical interface allows for only one.
Using scripts require more computer skills, and in particular more familiarity with the operating system.



\section{Translating into a script a processing launched via the graphical interface}
\label{section-console}

A simple way of writing a script that uses Unitex/GramLab is to launch the Unitex graphical interface,
to implement a preliminary version of the processing, and to translate your clicks into a sequence
of commands.
To do that, you can generate a log file of the launched operations (Section~\ref{section-log-file}).
You can also take advantage of the fact that the Unitex graphical interface keeps track
of these operations and can display them in the
console ("Info > Console", Section \ref{section-console}, Fig. \ref{fig-console}). You select the steps
to be retained, you copy them into a text file and if needed you adapt them into an operational script.
In order to select the steps in the console, click on the corresponding cells in the `Log \#' column
(a shift-click and a ctrl-click have the usual effect in selecting several items).
You can insert into the script programs or options documented in
Chapter~\ref{chap-external-programs}, even if they are not available via the interface.

\bigskip
\noindent You can formalize a script in shell or batch form (Section~\ref{section-batch}),
or in the form of a Unitex-interpreted script (Section~\ref{section-unitex-scripts}).



\section{Shell or batch scripts}
\label{section-batch}

You can put a script in the form of a shell or batch script and execute it in the command-line
interface of the operating system. For example, the following commands open the
\verb$80jours$ French corpus, without the REPLACE- or MERGE-mode preprocessings, but with the default
dictionary, \verb$Dela_fr$, and launch a "Locate pattern" with graph \verb$pattern.fst2$
and generate a concordance:

\begin{Verbatim}[fontsize=\small,fontfamily=helvetica]
mkdir "U:\Unitex\French\Corpus\80jours_snt" 
"C:\Unitex-GramLab\App\UnitexToolLogger.exe" Normalize
    "U:\Unitex\French/corpus.txt" "-rU:\Unitex\French\Norm.txt"
    "--output_offsets=U:\Unitex\French\Corpus\80jours_snt\normalize.out.offsets" -qutf8-no-bom
"C:\Unitex-GramLab\App\UnitexToolLogger.exe" Tokenize "U:\Unitex\French\Corpus\80jours.snt" 
    "-aU:\Unitex\French\Alphabet.txt" -qutf8-no-bom
"C:\Unitex-GramLab\App\UnitexToolLogger.exe" Dico "-tU:\Unitex\French\Corpus\80jours.snt" 
    "-aU:\Unitex\French\Alphabet.txt" "C:\Unitex-GramLab\French\Dela\Dela_fr.bin" -qutf8-no-bom
"C:\Unitex-GramLab\App\UnitexToolLogger.exe" SortTxt "U:\Unitex\French\Corpus\80jours_snt\dlf" 
    "-lU:\Unitex\French\Corpus\80jours_snt\dlf.n" "-oU:\Unitex\French\Alphabet_sort.txt"
    -qutf8-no-bom
"C:\Unitex-GramLab\App\UnitexToolLogger.exe" SortTxt "U:\Unitex\French\Corpus\80jours_snt\dlc" 
    "-lU:\Unitex\French\Corpus\80jours_snt\dlc.n" "-oU:\Unitex\French\Alphabet_sort.txt"
    -qutf8-no-bom
"C:\Unitex-GramLab\App\UnitexToolLogger.exe" SortTxt "U:\Unitex\French\Corpus\80jours_snt\err" 
    "-lU:\Unitex\French\Corpus\80jours_snt\err.n" "-oU:\Unitex\French\Alphabet_sort.txt"
    -qutf8-no-bom
"C:\Unitex-GramLab\App\UnitexToolLogger.exe" SortTxt
    "U:\Unitex\French\Corpus\80jours_snt\tags_err"
    "-lU:\Unitex\French\Corpus\80jours_snt\tags_err.n" "-oU:\Unitex\French\Alphabet_sort.txt"
    -qutf8-no-bom
"C:\Unitex-GramLab\App\UnitexToolLogger.exe" Locate "-tU:\Unitex\French\Corpus\80jours.snt" 
    "U:\Unitex\French\Graphs\pattern.fst2" "-aU:\Unitex\French\Alphabet.txt" -L -M --all -b -Y
    --stack_max=1000 --max_matches_per_subgraph=200 --max_matches_at_token_pos=400 
    --max_errors=50 -qutf8-no-bom
"C:\Unitex-GramLab\App\UnitexToolLogger.exe" Concord
    "U:\Unitex\French\Corpus\80jours_snt\concord.ind" "-fCourier new" -s10 -l40 -r55 --html 
    "-aU:\Unitex\French\Alphabet_sort.txt" --CL -qutf8-no-bom
\end{Verbatim}

\noindent For details of the scripts or any comments in it, you have to respect the operating system's
 conventions.



\section{Unitex/GramLab-interpreted scripts}
\label{section-unitex-scripts}\index{Package!\verbc{linguistic}}

Unitex/GramLab has also a scripting language, with several advantages over shell or batch
scripts.
\begin{itemize}
    \item Execution is faster than with an equivalent shell or batch script, since the interpreter manages a
virtual file system.
    \hyphenation{pac-kage}
    \item The script and all the resources it requires are encapsulated in a `linguistic package'
    that you can deploy in another environment later. This way, the development of the script and its
    use in a processing chain are completely independent:
\begin{itemize}
    \item you can deploy the package without any knowledge of Unitex or of the linguistic processing
    implemented in the script,
    \item the development environment and the deployment environment are not necessarily
    in the same operating system,
    \item the processing does not modify any data in the deployment environment, since it is applied
    to a copy of the corpus and uses a copy of the resources included in the package. Only the files
explicitly specified as being the output of the processing are copied into the deployment environment in the end 
of the processing.
\end{itemize}
\end{itemize}

\noindent The scripting language consists of the Unitex external programs, including
a few programs implemented specially for the script interpreter 
(Section~\ref{section-script-for-runscript}). There can be comment lines, beginning with 
the  \# character. The commands of the operating system are not recognised.

\bigskip\index{External programs!\verbc{RunScript}}\index{External programs!\verbc{BatchRunScript}}
\noindent The Unitex/GramLab script interpreter exists in two variants named \verb$RunScript$ and
\verb$BatchRunScript$. \verb$RunScript$ launches a script on a corpus contained in one or several files.
\verb$BatchRunScript$ processes a corpus consisting of all the files in a directory,
by launching the script once on each file, and it generates a separate output file for each.




\subsection{Implementing a linguistic package}
\label{section-packaga-creation}

A Unitex/GramLab linguistic package includes a script and all the language resources it uses,
except the input corpus. The package is organised in the form of a directory tree, and 
compressed in Zip format.

\bigskip
\noindent The tree root is a directory \verb$<ling_pkg_name>$ with two subdirectories
named \verb$script$ and \verb$resource$. The script must be placed in
 \verb$<ling_pkg_name>/script$ and all the required language resources in
\verb$<ling_pkg_name>/resource$. You can mimic in \verb$resource$ the organization of a
personal Unitex workspace, with a subdirectory for each language processed by the script
(for example \verb$French$), then, in each language directory, alphabet files
\verb$Alphabet.txt$ and \verb$Alphabet_sort.txt$, normalization file \verb$Norm.txt$ etc., and
subdirectories \verb$Dela$, \verb$Graphs$ etc.\ and, in turn, the respective data in them.
With this organization, it is easier to write the script by adapting a script from the console.

\bigskip
\noindent If you want the script to access the date of the package, create an empty text file in 
\verb$<ling_pkg_name>/resource$ and name it \verb$VERSION_AAAAMMJJ$
(without extension).

\bigskip
\noindent When the tree is finished, compress it in Zip format.

\bigskip
\noindent For example, in order to encapsulate the processing of Section~\ref{section-batch}
in a package named \verb$pkg.zip$, you have to copy into a \verb$pkg$ directory:
\begin{itemize}
\item in subdirectory \verb$resource$:
\begin{itemize}
\item the alphabet files, \verb$Alphabet.txt$ and \verb$Alphabet_sort.txt$, and the normalization
file \verb$Norm.txt$ in \verb$French$;
\item the dictionary files \verb$Dela_fr.bin$ and \verb$Dela_fr.inf$ in \verb$French/Dela$;
\item the graph \verb$pattern.fst2$ in \verb$French/Graphs$ (optionally, you can also copy the file
\verb$pattern.grf$ there);
\end{itemize}
\item in subdirectory \verb$script$, the script file.
\end{itemize}



\subsection{Launching with RunScript}
\label{section-runscript}\index{External programs!\verbc{RunScript}}
\index{External programs!\verbc{UnitexToolLogger}}
\verb$RunScript$ launches a script on a corpus contained in one or several files.
To do that, invoke \verb$UnitexToolLogger$ (Section~\ref{section-UnitexToolLogger})
with the following syntax:
\index{External programs!\verbc{SelectOutput}}
\index{External programs!\verbc{InstallLingResourcePackage}}

\begin{Verbatim}[fontsize=\small,fontfamily=helvetica]
UnitexToolLogger [ { SelectOutput <args1> } ] { InstallLingResourcePackage <args2> } { RunScript
    <args3> } { InstallLingResourcePackage <args4> }
\end{Verbatim}

\noindent Each pair of curly brackets delimits the invocation of a Unitex command with its arguments.

\begin{enumerate}
\item The \verb$SelectOutput$ step is optional but recommended. By including it in the following form:
\\
\verb$     { SelectOutput --output=off }$\\
you block the debugging messages of each subsequent command, which would consume space in the case of an extensive run.
\item The next step, \verb$InstallLingResourcePackage$, installs the content of the linguistic package
in the virtual file system.
\item The \verb$Rusncript$ step launches the script. It must set in  \verb$INPUT_FILE_1$ and
if necessary \verb$INPUT_FILE_2$, etc., the name of the file(s) that contain the input corpus; in
\verb$OUTPUT_FILE_1$, \verb$OUTPUT_FILE_2$, etc., the name of the output file(s);
in \verb$PACKAGE_DIR$ the name of the root directory of the linguistic package in the virtual
fils system; in \verb$CORPUS_WORK_DIR$ the name of the directory that will contain the files
generated during the execution of the script.
\item The last step writes the output file(s) and then terminates the virtual file system.
\end{enumerate}

\noindent For example, the following command uses the \verb$pkg.zip$ linguistic package of
Section~\ref{section-packaga-creation} and launches the 
 \verb$uniscript$ script contained in the package, with the \verb$80jours.txt$ file as input corpus
and \verb$80jours_result.html$ as output file.

\begin{Verbatim}[fontsize=\small,fontfamily=helvetica]
"C:\Unitex-GramLab\App\UnitexToolLogger.exe" { SelectOutput --output=off } {
    InstallLingResourcePackage -p U:\Scripts\pkg.zip -x "$:UnitexPkgResource" -v } { RunScript -v
    -a INPUT_FILE_1=U:\Unitex\French\Corpus\80jours.txt 
    -a "CORPUS_WORK_DIR=$:UnitexPkgWork" -a "PACKAGE_DIR=$:UnitexPkgResource"
    -a OUTPUT_FILE_1=U:\Unitex\French\Corpus\80jours_result.html 
    "$:UnitexPkgResource\script\uniscript" } { InstallLingResourcePackage -p U:\Scripts\pkg.zip 
    -x "$:UnitexPkgResource" -u -v }
\end{Verbatim}



\subsection{Launching with BatchRunScript}
\label{launch-script}\index{External programs!\verbc{BatchRunScript}}

\verb$BatchRunScript$ processes a corpus consisting of all the files in a directory.
It launches the script once on each file and generates a separate output file for each.
This way, you can implement massive processings efficiently, even on many documents
contained in separate files. The result is stored in an output directory, in files
with the same names as the input files, suffixed by \verb$.result.txt$.
\verb$BatchRunScript$ is multi-threaded and runs several instances in parallel: this way,
all the cores of the processing machine can work at the same time and process all the files faster.

\bigskip
\noindent To deploy a script with \verb$BatchRunScript$, use the following syntax:

\begin{verbatim}
UnitexToolLogger { BatchRunScript <args> }
\end{verbatim}

\noindent The arguments must set the name of the directory containing the input corpus
in the virtual file system, the name of the output directory, the number of threads, the name
of the linguistic package and the name of the script. For example, the following command processes
all the files in the \verb$input_folder$ directory, stores the output in \verb$output_folder$ and runs
on four threads:

\begin{Verbatim}[fontsize=\small,fontfamily=helvetica]
"C:\Unitex-GramLab\App\UnitexToolLogger.exe" { BatchRunScript -i .\input_folder -o .\output_folder 
    -t 4 .\pkg -v -p -m -s script/uniscript }
\end{Verbatim}

\noindent The \verb$-v$, \verb$-m$ and \verb$-p$ options control the information output on the
teminal during the processing. With \verb$-v -p$, you display the most messages, which is
useful during development. With \verb$-m$ and even more with \verb$-f$, you display the least
messages.

\bigskip
\noindent With \verb$BatchRunScript$, the running of a script on an input file cannot generate
several output files, as opposed to \verb$RunScript$.\footnote{You can bypass this constraint
with the \texttt{PackFile} and \texttt{UnpackFile} tools, by grouping several output files
into a single Zip file.}



\subsection{Developing a script for RunScript}
\label{section-script-for-runscript}
In order to write a script in a linguistic package, we recommend adapting a script from the
console, such as that of Section~\ref{section-batch}.

\begin{itemize}
\item In each line of the script from the console, remove the mention of the
\verb$UnitexToolLogger$ program at the beginning of the line, i.e.\ in our example
\verb$"C:\Unitex-GramLab\App\UnitexToolLogger.exe"$.
\item In the name of the resource files, substitute
\verb${PACKAGE_DIR}/resource$ for the path of the working directory.
In our example, the normalization file for French becomes
\verb${PACKAGE_DIR}/resource/French/Norm.txt$.
\item In the names of the files generated by the script, substitute
\verb${CURRENT_WORK_DIR}$ for the complete path. In our example, \verb$80jours.snt$
becomes \verb${CURRENT_WORK_DIR}/80jours.snt$.
\end{itemize}

\noindent The value of \verb$PACKAGEDIR$ is already set when the script begins to run, but
for \verb$CURRENT_WORK_DIR,$ a few lines are required at the beginning of the script
in order to name the directory, create it, copy the input corpus in it and make some
other arrangements:

\begin{Verbatim}[fontsize=\small,fontfamily=helvetica]
CURRENT_WORK_DIR = {CORPUS_WORK_DIR}/{UNIQUE_VALUE}
\end{Verbatim}

\noindent This line gives a name to \verb$CURRENT_WORK_DIR$. The value of
\verb$CORPUS_WORK_DIR$ is already set when the script begins to run.

\verb$UNIQUE_VALUE$ contient une chaine de caractères définie par \verb$RunScript$ ou par
\verb$BatchRunScript$, avec la garantie que la chaine est différente pour chaque exécution de
\verb$RunScript$ ou pour chaque thread de \verb$BatchRunScript$ s’exécutant simultanément. Cela
permet d'éviter toute collision entre fichiers temporaires pendant des travaux simultanés. Ensuite,
comme la commande \verb$mkdir$ n’est pas utilisable dans un script interprété par Unitex/GramLab, on
utilise \verb$DuplicateFile -p$ pour créer le répertoire:
\index{External programs!\verbc{DuplicateFile}}

\begin{Verbatim}[fontsize=\small,fontfamily=helvetica]
DuplicateFile -p {CURRENT_WORK_DIR}
\end{Verbatim}

\noindent On y copie le corpus d’entrée:

\begin{Verbatim}[fontsize=\small,fontfamily=helvetica]
DuplicateFile -i {INPUT_FILE_1} {CURRENT_WORK_DIR}/corpus.txt
\end{Verbatim}

\noindent Dans notre exemple, la première commande du script issu de la console utilisait la commande
\verb$mkdir$ pour créer le sous-répertoire \verb$corpus_snt$. On doit donc la remplacer par
\verb$DuplicateFile$:

\begin{Verbatim}[fontsize=\small,fontfamily=helvetica]
DuplicateFile --make-dir {CURRENT_WORK_DIR}/corpus_snt
\end{Verbatim}

\noindent Les deux lignes suivantes écrivent dans des fichiers des informations sur la version d'Unitex
utilisée:\index{External programs!\verbc{VersionInfo}}


\begin{Verbatim}[fontsize=\small,fontfamily=helvetica]
VersionInfo -n -o {CURRENT_WORK_DIR}/newrevision.txt
VersionInfo -s -o {CURRENT_WORK_DIR}/semver.txt
\end{Verbatim}

\noindent La ligne suivante sauvegarde la date de version du package linguistique, à condition d'avoir
inclus dans le package un fichier \verb$VERSION_AAAAMMJJ$ (voir section \ref{section-packaga-creation}):

\begin{Verbatim}[fontsize=\small,fontfamily=helvetica]
VersionInfo -B -o {CURRENT_WORK_DIR}/builddate.txt
\end{Verbatim}

\noindent Le corps du script peut apparaitre ensuite:

\begin{Verbatim}[fontsize=\small,fontfamily=helvetica]
Normalize "{CURRENT_WORK_DIR}/corpus.txt" "-r{PACKAGE_DIR}/resource/French/Norm.txt" 
    "--output_offsets={CURRENT_WORK_DIR}/corpus_snt/normalize.out.offsets" -qutf8-no-bom
Tokenize "{CURRENT_WORK_DIR}/corpus.snt" "-a{PACKAGE_DIR}/resource/French/Alphabet.txt"
    -qutf8-no-bom
Dico "-t{CURRENT_WORK_DIR}/corpus.snt" "-a{PACKAGE_DIR}/resource/French/Alphabet.txt" 
    "{PACKAGE_DIR}/resource/French/Dela/Dela_fr.bin" -qutf8-no-bom
SortTxt "{CURRENT_WORK_DIR}/corpus_snt/dlf" "-l{CURRENT_WORK_DIR}/corpus_snt/dlf.n" 
    "-o{PACKAGE_DIR}/resource/French/Alphabet_sort.txt" -qutf8-no-bom
SortTxt "{CURRENT_WORK_DIR}/corpus_snt/dlc" "-l{CURRENT_WORK_DIR}/corpus_snt/dlc.n" 
    "-o{PACKAGE_DIR}/resource/French/Alphabet_sort.txt" -qutf8-no-bom
SortTxt "{CURRENT_WORK_DIR}/corpus_snt/err" "-l{CURRENT_WORK_DIR}/corpus_snt/err.n" 
    "-o{PACKAGE_DIR}/resource/French/Alphabet_sort.txt" -qutf8-no-bom
SortTxt "{CURRENT_WORK_DIR}/corpus_snt/tags_err"
    "-l{CURRENT_WORK_DIR}/corpus_snt/tags_err.n" 
    "-o{PACKAGE_DIR}/resource/French/Alphabet_sort.txt" -qutf8-no-bom
Locate "-t{CURRENT_WORK_DIR}/corpus.snt"
    "{PACKAGE_DIR}/resource/French/Graphs/test.fst2" 
    "-a{PACKAGE_DIR}/resource/French/Alphabet.txt" -L -M --all -b -Y --stack_max=1000
    --max_matches_per_subgraph=200 --max_matches_at_token_pos=400 --max_errors=50
    -qutf8-no-bom
Concord "{CURRENT_WORK_DIR}/corpus_snt/concord.ind" "-fCourier new" -s10 -l40 -r55 --html 
    "-a{PACKAGE_DIR}/resource/French/Alphabet_sort.txt" --CL -qutf8-no-bom
\end{Verbatim}

\noindent La fin du script doit indiquer le nom du ou des fichiers résultats et libérer la mémoire:
\index{External programs!\verbc{DuplicateFile}}

\begin{Verbatim}[fontsize=\small,fontfamily=helvetica]
DuplicateFile -i {CURRENT_WORK_DIR}/corpus_snt/concord.html {OUTPUT_FILE_1}
DuplicateFile --recursive-delete {CURRENT_WORK_DIR}
\end{Verbatim}

\noindent On peut lancer le script avec la commande de la section~\ref{launch-script}. Le résultat
obtenu est juste un fichier de concordance \verb$80jours_result.html$, placé à côté de
\verb$80jours.txt$.

\bigskip
\noindent Si on ajoute au script, en avant-dernière position, la ligne:

\begin{Verbatim}[fontsize=\small,fontfamily=helvetica]
DuplicateFile -i {CURRENT_WORK_DIR}/corpus_snt/dlf {OUTPUT_FILE_2}
\end{Verbatim}

\noindent on doit aussi ajouter à la commande de lancement l'option:

\begin{Verbatim}[fontsize=\small,fontfamily=helvetica]
-a OUTPUT_FILE_2=U:\Unitex\French\Corpus\80jours_result.dic
\end{Verbatim}

\noindent On obtient alors comme résultat un fichier dictionnaire \verb$80jours_result.dic$, placé à côté
des deux autres.
