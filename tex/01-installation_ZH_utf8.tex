\chapter{安装Unitex}
\label{chap-install}
Unitex是一个多平台软件,它能在Windows、Linux或MacOS下良好运行。这一章节描述了Unitex在这些操作系统下的安装和运行。这一章节也阐述了加入新语言和卸载的方法。

\section{证书}
\label{section-licences}
\index{LGPL证书}\index{LGPL证书}
Unitex是一个开放的软件,也就是说这些工程的资源是由该软件分配的,而且每个人都能修改或再分配。Unitex工程的代码是遵循LPGL证书的(\cite{LGPL}),除了对常规表达的操作库TRE(\cite{TRE}),这遵循着证书"2 Clause BSD" (比LPGL证书更加宽松),而LibYAML库遵循的MIT证书也比LGPL更宽松。
LGPL证书比GPL宽松,因为它能在不开源的软件中使用LGPL代码。从用户们的角度两者没有区别,因为他们都能自由的使用和分配。

\bigskip
\noindent所有Unitex分配的语言学暑假都遵循LGPLLR证书
\index{证书LGPLLR} (\cite{LGPLLR}).

\bigskip
\noindent GPL,LGPL和LGPLLR证书的完整文本都在用户手册的附件中。

\section{执行Java的环境}
Unitex包含了由Java语言编写的图形界面而外部程序由 \textit{C/C\kern-.05em\raisebox{.5ex}{++}\kern-.1em}。这种编程语言的混合使用使程序更迅速并能在多个操作系统下运行。


\bigskip
\noindent 为了使用图形界面,需要先准备一个虚拟机\index{Java虚拟机}或
JRE\index{JRE} (Java运行环境\index{Java运行环境}\index{Java!JRE}).

\bigskip
\noindent 在图形模式下运行,Unitex需要1.6(或更高版本)的Java版本。如果你还是        Java的旧版本,Unitex将你选择了你的工作语言后锁定。

\bigskip
\noindent 您可以随意下载的虚拟机操作系统从Sun Microsystems的网站 (\cite{site-java}) : 
\url{http://java.sun.com}.

\bigskip
\noindent 如果你正在使用Linux或MacOS或如果你使用一个版本的Windows的管理做实个人账户的用户,你需要问你的系统管理员安装Java。



\section{Windows下安装}
\index{Installation!sous Windows}
如果你希望在多用户的Windows计算机上安装Unitex,最好是让您的管理员这样做。如果你是你的机器的唯一用户,您可以自己安装。



\section{Windows安装}
\index{Windows环境下的安装}
    如果您希望在Windows多用户系统上安装Unitex,需要管理员权限。如果您是该计算机的唯一用户,您可以自己安装。

\bigskip
\noindent 解压缩文件 \index{文件!\verb+Unitex3.1beta.zip+} \verb+Unitex3.1beta.zip+ (或 \verb+Unitex3.0.zip+)
--- 您可以在以下地址下载这些文件: \url{http://igm.univ-mlv.fr/~unitex} ---
在提前创建的\verb+Unitex3.1beta+目录(文件夹)中,
在偏好目录\verb+Program Files+,而它会调用系统目录中的Unitex.\index{系统目录Unitex}\index{文件夹|见{目录}}

\bigskip
\noindent 解压缩之后,目录 \verb+Unitex3.1beta+
(le répertoire système Unitex) 包含了多个名为\verb+App+的子目录。 最后一个目录包含了名为\verb+Unitex.jar+\index{文件!\verb+Unitex.jar+}的文件。
 该Java文件用于运行图形界面。您仔细双击该图表就能运行该软件。
为了更方便地运行,建议您在桌面建立快捷方式。

\section{Linux安装}
\index{Linux环境下的安装}
在Linux和MacOS系统上安装Unitex,要求管理员权限。您将文件\verb+Unitex3.1beta.zip+解压缩于名为\verb+Unitex+的文件夹,通过以下命令:


\bigskip \noindent \verb$unzip Unitex3.1beta.zip -d Unitex$

\bigskip
\noindent 该目录将会调用系统目录Unitex\index{系统目录Unitex},
接着您在目录 \verb|Unitex/Src/C++/build| 中通过以下命令进行编译:


\bigskip \verb+安装+

\bigskip
\noindent 如果是64位计算机:
 
\bigskip \verb+make install 64BITS=yes+
\bigskip
\noindent 然后在接下来的模型中创建一个别名在:

\bigskip \verb$alias unitex='cd /..../Unitex/App/ ; java -jar Unitex.jar'$


\section{MacOS X系统安装Unitex}
\index{MacOS X系统安装Unitex}
\label{section-macos-install}
\noindent 注:这个简短的教程将告诉你如何在Mac OS X系统中 安装和运行Unitex。
欢迎提出你的问题或对我们的评价和建议。
\noindent 请联系: \url{cedrick.fairon@uclouvain.be}


\bigskip
\noindent Java 的 Oracle正式版 适用于 MacOS X 10.7.3 (Lion)至最新版本。
	查看``Mac OS X系统安装和使用Oracle Java的基本配置及详细说明'' 于以下网页 \url{https://www.java.com/fr/download/faq/java_mac.xml}

	

\bigskip
\noindent 适用于 MacOS X 10.7 至最新版本 的 Apple的Java结构,
	详情见以下网页 \url{https://support.apple.com/kb/DL1572}. 
	适用于 MacOS X 10.6 的 Apple的结构,
	详情见以下网页 \url{https://support.apple.com/kb/DL1573}.

\bigskip
\noindent Java 1.6 正式版 适用于MacOS X 10.5, 64-bit Intel (Core 2 Duo), 但对于OS X的更早版本(10.4或更早)没有正式的解决方法,
PowerPC 和 32-bit Intel (Core Duo)。然而,如果你的系统是OS X 10.5( MacOS 64-bit Intel),只需有 JRE 1.6. Apple。唯一的问题是这个版本不能默认启动。
	查看 ``Java 适用于 Mac OS X 10.5 Update 10'', 于以下网页 \url{https://support.apple.com/kb/DL1359}


\noindent\textbf{如何查看处理器是32位还是64位?}

\noindent Apple菜单中, 点击 "关于本机"。你将看到以下条目:
"Processor : x.xx Ghz Intel Core Duo", 您的处理器是 32 bits.

\bigskip
\noindent 如果你看到:"Processeur: x.xx Ghz Intel Core 2 Duo", 或你的处理器是Intel(如Xeon)的, 那么你的处理器是64位的。

\subsection{使用 Apple Java 1.6 runtime}
\bigskip\index{Java!Apple Java 1.6 runtime}
\noindent 如果你使用 Mac OS X 10.5 (或更高版本) 于Intel 64 bits处理器, 你就能轻松地使用Java 1.6 d'Apple。
详细说明于以下网页 \url{https://support.apple.com/kb/DL1359}.

\noindent 点击进入 Application -> Utilities -> Java Preferences 
然后在 "Java Applications"列表中 找到"Java SE 6"。

\subsubsection{选项1:更改Java应用程序的默认运行}
\noindent 如果你不使用Java 1.5应用程序,
那么你可以按照个人使用习惯,在"Applications Java"列表中将"Java SE 6"置顶。

\subsubsection{选项2:创建一个别名运行Java1.6}
\noindent 如果你不想修改Java中的全局变量,你能创建一个别名。

\bigskip
\noindent \verb+alias jre6="/System/Library/Frameworks/JavaVM.framework/Versions/+
\noindent \verb+1.6/Commands/java"+
   
\bigskip
\noindent \verb+jre6 -jar Unitex.jar+

\bigskip
\noindent 然后从终端运行Unitex

%\subsection{SoyLatte}

%\subsection{Comment compiler les programmes les C++ Unitex sur un ordinateur Macintosh}


\subsection{如何使Mac OS中的所有文件可见}
\noindent 查看
\url{http://www.macworld.com/article/51830/2006/07/showallfinder.html}.

\bigskip
\noindent 或立即尝试... 输入: 

\bigskip
\verb+defaults write com.apple.Finder AppleShowAllFiles ON+

\bigskip
\noindent 然后重新运行Finder:

\bigskip
\verb+killall Finder+

\begin{figure}[!h]
\begin{center}
\includegraphics[width=12cm]{resources/img/fig-mac6.png}
\caption{重新运行 Finder\label{fig-mac6}}
\end{center}
\end{figure}

\bigskip
\noindent 要返回原来的配置,需要输入:

\bigskip
\verb+defaults write com.apple.Finder AppleShowAllFiles OFF+


\section{第一次使用}
如果您使用的是Windows,你需要选择一个工作主目录
\index{Répertoire!personnel de travail}。 您可以在“信息>首选项...>目录”更改。要创建一个文件夹,点击文件夹图标
(见图 \ref{fig-creation-personal-directory}).

\bigskip
\noindent 如果您使用的是Linux和Mac OS,该程序会自动创建一个个人工作目录,称为\verb+/unitex+ 在目录\verb+$HOME+. 

\bigskip
\noindent 在个人工作目录,您将存储您的Unitex个人资料。您所使用的每种语言,程序会复制语言的树在你的工作目录中,除了字典。然后,您可以编辑,你需要不损坏的复制你的数据存放在Unitex系统目录系统数据。\index{Répertoire!système Unitex}


\begin{figure}[h]
\begin{center}
\includegraphics[width=6.3cm]{resources/img/fig1-1.png}
\caption{Windows系统第一次使用}
\end{center}
\end{figure}

\begin{figure}[h]
\begin{center}
\includegraphics[width=7cm]{resources/img/fig1-2.png}
\caption{Linux系统第一次使用}
\end{center}
\end{figure}

\begin{figure}[h]
\begin{center}
\includegraphics[width=13cm]{resources/img/fig1-3.png}
\caption{个人工作目录的建立
\label{fig-creation-personal-directory}}
\end{center}
\end{figure}



\section{添加新的语言}
\index{Ajout de nouvelles langues}

\bigskip
\noindent 有两种方法来添加语言。如果你想添加一个新的语言到所有用户,你必须复制进Unitex系统语言目录,\index{Répertoire!système Unitex}
这需要有访问该目录的权利(你可能需要询问系统管理员).但是,如果是这语言有关的唯一用户,你可以在你的工作目录复制目录。\index{Répertoire!personnel de travail}您将能够在这语言工作而不提供给其他用户。


\section{卸载}
不管是什么系统,你的工作,你只需删除\verb+Unitex目录所有的系统文件。在Windows上,则必须删除快捷方式\verb+Unitex.jar+ \index{Fichier!\verb+Unitex.jar+} 如果您已创建了一个;同样在Linux或MacOS上,如果你已经创建了一个别名。


\section{Unitex开发}
\label{section-unitex-developpers}

如果你是一个程序员,那你可能感兴趣是Unitex的C++代码来源的链接。为了推动这项工作,你就可以编译Unitex为包含除库\verb+main+s所有功能以外Unitex功
能的动态.\url{http://docs.unitexgramlab.org/projects/unitex-library/fr/latest/}页面包含有关库文档。


\bigskip
Linux/MacOS下, 输入:

\bigskip
\verb+make LIBRARY=yes+

\bigskip
\noindent 你会得到一个库命名为 \verb+libunitex.so+. 如果你想制作的Windows DLL文件名为\verb+unitex.dll+,请使用以下命令:

\bigskip
Windows: \verb+make SYSTEM=windows LIBRARY=yes+

带mingw32的交叉编译: \verb+make SYSTEM=mingw32 LIBRARY=yes+

\bigskip
\noindent 在所有情况下,你也将获得一个所谓的AA程序
\verb+Test_lib+(\verb+.exe+). 
如果一切行之有效,该程序应显示以下画面:

\begin{verbatim}
Expression converted.
Reg2Grf exit code: 0

#Unigraph
SIZE 1313 950
FONT Times New Roman:  12
OFONT Times New Roman:B 12
BCOLOR 16777215
FCOLOR 0
ACOLOR 12632256
SCOLOR 16711680
CCOLOR 255
DBOXES y
DFRAME y
DDATE y
DFILE y
DDIR y
DRIG n
DRST n
FITS 100
PORIENT L
#
7
"<E>" 100 100 1 5
"" 100 100 0
"a" 100 100 1 6
"b" 100 100 1 4
"c" 100 100 1 6
"<E>" 100 100 2 2 3
"<E>" 100 100 1 1
\end{verbatim}
