\chapter*{Introduction}
\addcontentsline{toc}{chapter}{Introduction}

Unitex est un ensemble de logiciels permettant de traiter des textes en langues naturelles en
utilisant des ressources linguistiques. Ces ressources se présentent sous la forme de dictionnaires
électroniques, de grammaires et de tables de lexique-grammaire. Elles sont issues de travaux initiés
sur le français par Maurice Gross au Laboratoire d’Automatique Documentaire et Linguistique (LADL)
. \index{LADL} Ces travaux ont été étendus à d’autres langues au travers du réseau de laboratoires
RELEX.

\bigskip
\noindent Les dictionnaires électroniques décrivent les mots simples et composés d’une langue en
leur associant un lemme ainsi qu’une série de codes grammaticaux, sémantiques et flexionnels. La présence de ces dictionnaires constitue une différence majeure par rapport aux outils
usuels de recherche de motifs, car on peut faire référence aux informations qu’ils contiennent
et ainsi décrire de larges classes de mots avec des motifs très simples. Ces dictionnaires sont
représentés selon le formalisme DELA et ont été élaborés par des équipes de linguistes pour
plusieurs langues (français, anglais, grec, italien, espagnol, allemand, thaï, coréen, polonais,
norvégien, portugais, etc...).


\bigskip
\noindent Les grammaires sont des représentations de phénomènes linguistiques par réseaux de
transitions récursifs (RTN), un formalisme proche de celui des automates à états finis. De
nombreuses études ont mis en évidence l’adéquation des automates aux problèmes linguistiques et ce,
aussi bien en morphologie qu’en syntaxe ou en phonétique. Les grammaires manipulées par Unitex
reprennent ce principe, tout en reposant sur un formalisme encore plus puissant que les automates.
Ces grammaires sont représentées au moyen de graphes que l’utilisateur peut aisément créer et
mettre à jour.

\bigskip
\noindent Les tables de lexique-grammaire sont des matrices décrivant les propriétés de certains
mots. De telles tables ont été élaborées pour tous les verbes simples du français dont elles
décrivent les propriétés syntaxiques. L’expérience ayant montré que chaque mot a un comportement
quasi unique, ces tables permettent de donner la grammaire de chaque élément de lexique, d’où le nom
de lexique-grammaire. Unitex permet de construire des grammaires à partir de telles tables.

\bigskip
\noindent Unitex est un moteur permettant d’exploiter ces ressources linguistiques. Ses
caractéristiques techniques sont la portabilité, la modularité, la possibilité de gérer des langues
possédant des systèmes d’écritures particuliers comme certaines langues asiatiques et l’ouverture,
grâce à une distribution en logiciel libre. Ses caractéristiques linguistiques sont celles
qui ont motivé l’élaboration des ressources : la précision, l’exhaustivité et la prise en compte
des phénomènes de figement, notamment en ce qui concerne le recensement des mots com-
posés.




\section*{Quoi de neuf depuis la version 3.1~?}
\addcontentsline{toc}{section}{Quoi de neuf depuis la version 3.1~?}
Voici les principales nouvelles fonctionnalités:
\begin{itemize}
  \item Unitex/GramLab inclut maintenant des données expérimentales sur le chinois.
  \item On peut ouvrir un sous-graphe --- ou le créer s'il n'existe pas --- en faisant un clic droit sur un
  appel à ce sous-graphe.
  \item Le développement fait appel à l'intégration continue; le code source, les ressources
  linguistiques et les versions précédentes sont disponibles sur des dépôts de données publics:\\
  \verb$          https://github.com/UnitexGramLab$
  \item CasSys et les graphes de généralisation d'étiquetage (chapitre~\ref{chap-cassys}) sont plus
  satisfaisants.
  \item L'exploration des chemins des grammaires (section~\ref{explore-paths}) fonctionne mieux.
  \item Il y a une meilleure compatibilité entre le mode morphologique, l'option de recherche de motifs
  "All matches",   le mode debug, la recherche de motifs dans l'automate du texte
  (section~\ref{section-locate-tfst}) et les variables d'entrée de dictionnaire.
  \item Il y a une fonctionnalité "rechercher et remplacer" pour les graphes et pour l'automate du texte;
  les fonctionnalités "défaire" et "refaire" fonctionnent maintenant aussi sur l'automate du texte.
  \item On peut sauvegarder un corpus traité dans le répertoire qu'on veut; on peut sauvegarder
  et ouvrir des concordances.
  \item L'automate du texte s'affiche avec un marquage en couleur plus satisfaisant.
  \item Les dictionnaires du français ont été mis à jour. Les orthographes anciennes ont été retirées du
  dictionnaire du portugais du Brésil conforme à l'orthographe actuelle. Les graphes d'affixes verbaux
  du malgache ont été mis à jour.
  \item Des graphes pour les nombres en toutes lettres dans cinq langues sont fournis.
  \item Le menu pour les dictionnaires a une liste des dictionnaires qui ont été ouverts récemment.
  \item Le menu "File edition" ouvre un répertoire adéquat en fonction du type de fichier.
  \item La compatibilité avec les corpus en XML est améliorée.
  \item L'ordre alphabétique en grec moderne est pris en charge.
  \item Les codes sémantiques dupliqués sont détectés dans les dictionnaires de lemmes polylexicaux
  aussi.
  \item Le format tabulaire de l'automate du texte a un bouton pour filtrer les étiquettes.
  \item Les boites de dialogue se ferment quand on appuie sur la touche échappement.
  \item Les guillemets vides (\verb$""$) sont autorisés dans les graphes et interprétés comme
  \verb$<E>$.
  \item Le concordancier peut maintenant traiter des textes avec des mots de plus de
  5~000~caractères.
  \item Les boites de dialogue affichant un message d'erreur ont un bouton "Copy". La liste des
  sous-graphes appelés par un graphe a aussi un bouton de copie.
  \item On peut configurer une option dabs les préférences pour que les messages d'erreur
  n'apparaissent pas.
  \item Des fonctionnalités récentes ont été documentées dans le manuel d'utilisation.
  \item Diverses anomalies ont été corrigées. 
\end{itemize}

\noindent Merci à Cristian Martinez, Marwin Damis, Anubhav Gupta, Renaud Mouronval,
Maxime Petit, Victorien Villiers et Gilles Vollant pour leurs contributions substantielles.

\clearpage

\section*{Contenu}
\addcontentsline{toc}{section}{Content}
\noindent Le chapitre \ref{chap-install} décrit comment installer et lancer Unitex.

\bigskip \noindent Le chapitre \ref{chap-text} présente les différentes étapes de l'analyse d'un
texte.

\bigskip \noindent Le chapitre \ref{chap-dictionaries} décrit le formalisme 
des dictionnaires électroniques DELA et les différentes opérations qui peuvent leur être appliqués.

\bigskip \noindent Les chapitres \ref{chap-regexp} et \ref{chap-grammars}
présentent les différents moyens d’effectuer des recherches de motifs dans des textes.
Le chapitre \ref{chap-grammars} décrit en détail l'utilisation de l'éditeur de graphe.

\bigskip \noindent Le chapitre \ref{chap-advanced-grammars} est consacré aux différentes
utilisations possibles des grammaires. Les particularités de chaque type de grammaires y sont
présentées.

\bigskip \noindent Le chapitre \ref{chap-text-automaton} présente le concept d'automate du texte 
et décrit les propriétés de cette notion. Ce chapitre  décrit également les opérations sur cet
objet, en particulier, comment désambiguiser les items lexicaux avec le programme ELAG.

\bigskip \noindent Le chapitre \ref{chap-lexicon-grammar} contient une présentation des tables du
lexique-grammaire, et la description d'une méthode de construction de  grammaires fondées sur ces
tables.

\bigskip \noindent Le chapitre \ref{chap-alignment} décrit le module d'alignement de texte basé sur
l'outil XAlign.

\bigskip \noindent Le chapitre \ref{chap-multiflex} décrit le module de flexion des mots composés,
en tant que complément du système de flexion des mots simples, présenté au chapitre
\ref{chap-dictionaries}.

\bigskip \noindent Le chapitre \ref{chap-cassys} décrit le système de cascades de transducteurs
CasSys.

\bigskip \noindent Le chapitre \ref{chap-scripts} explique comment utiliser Unitex/GramLab
par l'intermédiaire de scripts qui lancent des programmes.

\bigskip \noindent Le chapitre \ref{chap-external-programs} contient une description détaillée des
programmes externes qui composent le système Unitex.

\bigskip \noindent Le chapitre \ref{chap-file-formats} contient une description de tous les formats
de fichiers utilisés par Unitex.


\bigskip \noindent Le lecteur trouvera en annexe la licence LGPL sous  laquelle le code source
Unitex est diffusé, ainsi que la licence LGPLLR qui s'applique pour les données linguistiques
distribuées avec Unitex. Il y trouvera aussi la licence 2-clause BSD qui s'applique à la
bibliothèque TRE, utilisée par Unitex pour les filtres morphologiques.

\clearpage

\section*{Contributions à Unitex}
\addcontentsline{toc}{section}{Unitex contributeurs}
Unitex est né comme un pari sur la puissance de la philosophie Open Source dans le monde
universitaire (voir \url{http://unitexgramlab.org/why-unitex}), en s'appuyant sur l'hypothèse que les gens seraient intéressés à partager leurs connaissances et leurs compétences dans un tel projet ouvert.
%The following list sounds like Open Source is good for science:

\begin{itemize}                   
    \item Olivier Blanc: a intégré le système ELAG à Unitex, originellement conçu par Eric Laporte,
    Anne Monceaux et certains de leurs étudiants, a également écrit \verb+RebuildTfst+ (anciennement
     appelé \verb+MergeTextAutomaton+)
    \item Matthieu Constant: auteur de \verb+Grf2Fst2+
    \item Julien Decreton: auteur de l'éditeur de texte intégré à Unitex,
    	    a aussi réalisé la fonctionnalité \verb+undo+ de l'éditeur de graphe
    \item Marwin Damis: amélioration de l'interface de l'automate du texte
    \item Claude Devis: ajout des filtres morphologiques, fondé sur la bibliothèque TRE
    \item Nathalie Friburger: auteure de \verb+CasSys+
    \item Anubhav Gupta: a perfectionné \verb+CasSys+
    \item Hyun-Gue Huh: auteur de l'outil de génération de dictionnaires coréens
    \item Claude Martineau: a travaillé sur la flexion des mots simples dans \verb+MultiFlex+
    \item Cristian Martinez: a mis en place la chaine d'intégration continue et corrigé des anomalies
    importantes
    \item  Renaud Mouronval: a amélioré l'exploration des chemins des grammaires
    \item Sebastian Nagel: a optimisé de nombreuses parties du code, il a également adapté
    	    \verb+PolyLex+ pour l'allemand et le russe
    \item Alexis Neme: a optimisé \verb+Dico+ et \verb+Tokenize+, a aussi intégré \verb+Locate+
    dans \verb+Dico+ pour accepter des graphes dictionnaires
     \item Aljosa Obuljen: auteur de \verb+Stats+
     \item Sébastien Paumier: développeur principal
    \item Maxime Petit: a amélioré la fonctionnalité "rechercher et remplacer" pour les graphes
     \item Agata Savary: auteure de \verb+MultiFlex+
    \item Anthony Sigogne: auteur de \verb+Tagger+ et de \verb+TrainingTagger+
    \item Gilles Vollant: auteur de \verb+UnitexTool+, a optimisé beaucoup
    	    d'aspects du code d'Unitex (mémoire, vitesse, compatibilité multi-compilateur, etc.) et corrigé
    	    d'innombrables anomalies
    \item Patrick Watrin: auteur de \verb+XMLizer+, a travaillé sur l'intégration de \verb+XAlign+ à Unitex
    \item Anthony Sigogne: auteur de \verb+Tagger+ et de \verb+TrainingTagger+
\end{itemize}

\bigskip
\noindent Il faut ajouter que Unitex serait inutile sans les précieuses ressources linguistiques
qu'il renferme. Toutes ces ressources sont le fruit d'un énorme et difficile travail effectué par
des personnes qui ne doivent pas être oubliées. Certaines sont citées dans les avertissements qui
sont fournis avec les dictionnaires, une information complète est disponible sur:

\bigskip
\noindent \url{http://unitexgramlab.org/language-resources}


\section*{Si vous utilisez Unitex dans des projets de recherche...}
\addcontentsline{toc}{section}{Si vous utilisez Unitex dans des projets de recherche...}
Unitex a été utlisé dans plusieurs projets de recherche. Certains sont listés dans la section 
``Related works'' de la page d'accueil d'Unitex. Si vous avez effectuer des travaux de recherche
avec Unitex (ressources, projet, article, thèse, ...) et si vous désirez qu'ils soient référencés
sur le site envoyez un mail à \url{unitex-devel@univ-mlv.fr}.

